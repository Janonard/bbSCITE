\section{Motivation}

Cancer is a widespread and often lethal disease\cite{10.1001/jamaoncol.2021.6987} where body cells mutate in a way that increases their cell division speed as well as their lifetime, while also evading immune responses. There are many possible treatment methods like surgery, chemotherapy, or radiation therapy, but their effectiveness often depends on the exact type of tumor. Tumors however are heterogenous: Individual tumor cells may mutate again and form new subclones that compete against others \cite{nik2012life}. Treating the dominant subclones however seems to provide an advantage for previously minor, resistant subclones \cite{gillies2012evolutionary}. Therefore, knowledge of existing subclones and their evolutionary history may help future treatments \cite{greaves2012clonal, stratton2009cancer, swanton2012intratumor}.

Bulk sequencing of tumor cells and analysing the found mutations appears to be already widely used, but seems to miss smaller, upcoming variants that are averaged out in the mass of cells \cite{navin2014cancer}. Therefore, research has been done to utilise single cell sequencing. With this technique, the exact genome of individual cells can be identified and compared against other cells. However, it also comes with high error rates and parts of a cell's genome are often lost during the process \cite{tree2016}. It is therefore very hard to identify which subclones actually exist and how they are related. There is therefore a need for algorithms that compute the most-likely mutation history. Such an algorithm is \acs{SCITE}\acused{SCITE} \cite{tree2016}, which is short for \acl{SCITE}.

\acp{FPGA} are computer chips that contain a lattice of logic, computation, and memory units that are connected via programmable connections. They can be used like complete, reprogrammable computer chips and have already been used extensively for chip prototyping and verification \cite{rodriguez2007features}. In recent years they have also become interesting as computation accelerators for \ac{HPC} users due to their low power consumption compared to their performance \cite{betkaoui2010comparing}. However, developing efficient \ac{FPGA} designs is often tedious since compiling them may take multiple hours and naive designs are often multiple orders of magnitudes worse than optimized ones \cite{betkaoui2010comparing}.

The original authors of \ac{SCITE} provided a functional, but rather unoptimized implementation of their algorithm. There is however an unpublished report by Dominik Ernst et al. \cite{ernst2020Performance} about their efforts to optimize the algorithm for parallel CPU architectures. Their list of parallelizable subtasks also contained entries that can be well exploited on \acp{FPGA}. Our general goal for this thesis is therefore to exploit these opportunities and develop an efficient implementation of \ac{SCITE} for \acp{FPGA}. 

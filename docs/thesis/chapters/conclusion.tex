\chapter{Conclusion}
\label{ch:conclusion}

\ac{ffSCITE} reaches the goal of being an implementation of the \ac{SCITE} algorithm for \acp{FPGA} with higher throughput than the reference implementation by Jahn et al. \cite{tree2016}. This algorithm computes the most likely mutation history of a group of tumor cells, given noisy information about their mutation status.

Our biggest contribution is an improvement to the mutation tree encoding: The reference implementation uses parent vectors as the canonical data structure, which contain the parent of every node in the tree. However, it also uses ancestor matrices to analyze the mutation tree, which denote whether there is a path from one node in the tree to another. These ancestor matrices are constructed from parent vectors and this operation is not practical on \acp{FPGA}. We, therefore, developed algorithms to reconstruct the mutation tree from an ancestor matrix and to compute the resulting ancestor matrix of a tree modification based on the old one. These algorithms are impractical on \acp{CPU} but work well on \acp{FPGA} due to their flexibility regarding custom logic.

Additionally, we reformulated the algorithm to compute a mutation tree's likelihood: The reference implementation uses two different implementations with different precisions to improve performance, but since \ac{ffSCITE} would not benefit from computing it twice, we based our formulation on the fast and less-precise implementation. The likelihood computation algorithm can be described as a mapping step followed by a two-dimensional reduction. The original formulation executed the reductions together with the mapping operation, which created too many loop-carried dependencies and resulted in a structurally weak design. We, therefore, split the mapping and the two reductions into separate loops and used loop unrolling to improve the loop's throughput at the cost of higher resource usage.

The previous optimizations resulted in a macro-pipeline where different components work independently from one another to execute a chain step. This pipeline needs to be filled at all times to maximize the mean throughput. We did this by filling the pipeline with initial chain states once, feeding the emitted states back after a chain step, and replacing them with new initial states once the chains are complete. Our scheme ensures that the pipeline is always filled and also simplifies and minimizes interactions with off-chip memory. This reduces compilation times and potentially increases energy efficiency.

Lastly, we also dealt with issues regarding random number generation: We used \acfp{URNG} and distribution algorithms from the C++ standard library and Intel's oneDPL to reduce development time. While these implementations work correctly on \acp{FPGA}, they are originally designed for \acp{CPU}. We especially faced issues with our \ac{URNG}, the Minstd \ac{URNG} \cite{park1988random}, since it uses 64-bit integer multiply and remainder operations. These operations have a high resource usage and latency and since their output is supposed to appear random, they create a data dependency that leads to an increased \acf{II} for the loop that uses them. Therefore, We isolated the random draws into a separate kernel and reduced the \ac{II} of its main loop to a level below the current bottleneck. Finding and implementing a better system for random draws remains an open direction.

The resulting chip design utilizes up to 69\% of the resources available on our target device, a Bittware 520N card with an Intel Stratix 10 GX 2800 \ac{FPGA}. Our design achieves a constant throughput of 577.8 thousand chain steps per second and is bottlenecked by the mapping step of the likelihood computation, as we have predicted based on the synthesis report. Compared to the reference implementation run on one of Noctua 2's nodes, which have AMD Milan 7763 \acp{CPU}, our implementation has up to 8.6 times higher throughput.

\section{Open directions}
\label{sec:open_directions}

\todo[inline]{Note that time was lacking}

\begin{itemize}
    \item Feature: Homocygous/Heterocygous mutations
    \item Feature: Sum score
    \item Feature: Finding the most-likely beta
    \item Feature: Co-optimal trees
    \item Feature: Sampling from posterior distribution
    \item New Feature: Continuous likelihood outputs
    \item New Feature/Optimization: Multi-Input
    \item Optimization: Customized RNG and distributions (Lower resource usage, increase throughput)
    \item Optimization: Replicate MCMC step (increase throughput, requires pipelined chain step loop)
    \item Optimization: Transpose ancestor matrix (simplify memory access pattern, increase scalability)
    \item Exploration: Energy efficiency
\end{itemize}
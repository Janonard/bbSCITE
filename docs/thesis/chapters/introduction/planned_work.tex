\section{Goals and Planned Work}

With the algorithm introduced and possible performance problems identified, we can formulate our goals: We plan to implement \ac{SCITE} for the Intel Stratix 10 GX 2800 \acp{FPGA} of the Noctua supercomputers at the Paderborn University using the Intel oneAPI/DPC++/SYCL toolchain, since we have the most experience with this target and toolchain. We want to maximize the number of markov chain steps executed per second, what we call henceforth call throughput, while still producing results of similar likelihood to the original reference implementation. We want to achieve at least as much throughput as the original reference implementation and if possible, we also want to achieve more throughput than the implementation of Ernst et al. \cite{ernst2020Performance}. This is however an optional goal since there is a possibility that we encounter optimization problems that can not be resolved in time and since we do not have access to the implementation or its performance figures yet. We also want to develop benchmarking and result validation tools for our implementation. This requires a performance model that can accurately predict and explain the runtime of the design given the input dimensions and performance parameters, as well as a hypothesis test framework to check whether our implementation's quality matches those of the reference implementation. Lastly, we also want to try to identify the changes required to adapt our implementation to \ac{infSCITE} and to evaluate whether it is feasible. However, this is also an optional goal since it is not essential to our main goal of performance engineering. Figure \ref{fig:goals} contains a summarized list of goals.

\begin{figure}
    \begin{itemize}
        \item Implement \ac{SCITE} for \acp{FPGA}
        \begin{itemize}
            \item Maximize throughput (number of markov chain steps executed per second)
            \item Produce solutions of similar likelihood as the original reference implementation
            \item Achieve at least as much throughput as the original reference implementation
            \item (Optional) Achieve more throughput than the implementation by Ernst et al. \cite{ernst2020Performance}.
        \end{itemize}
        \item Develop benchmarking and validation tools
        \begin{itemize}
            \item Develop a performance model for runtime prediction and explanation
            \item Develop a hypothesis test framework for the result quality
        \end{itemize}
        \item (Optional) Identify and evaluate changes required for \ac{infSCITE} adaptation
    \end{itemize}

    \centering
    \caption{Summary/Overview of the goals of the thesis}
    \label{fig:goals}
\end{figure}

In order to fullfil these goals, we will start with an initial implementation of the algorithm. This implementation may be naive and therefore possibly inefficient, but it should be functional and correct. Parallel to that, we will start to set up the verification and benchmarking system to establish the baseline of the initial FPGA implementation and the reference implementation. Once the foundations are established, we can begin to optimize. There is not much sense in planning this task since it heavily depends on the bottlenecks at hand, but it will probably result in repeated iterations of profiling, identification of bottlenecks, research in existing solutions, implementing existing or original ideas and analyzing the changes. We will write-up the findings from all analysis stages, which will aid use in the last phase: Writing the thesis and proofreading it. There, we will describe what performance we could achieve, which bottlenecks we have encountered, how we have resolved them, and how our decisions have affected the adaptability to \ac{infSCITE}. More precisely, we plan to use the structure in Figure \ref{fig:thesisstructure} for the thesis.

\begin{figure}
    \begin{itemize}
        \item Introduction
        \begin{itemize}
            \item Motivation
            \item Background
            \item Related Work
        \end{itemize}
        \item Implementation
        \begin{itemize}
            \item Notable Bottlenecks and Solutions
            \item Adaptability of \ac{infSCITE}
            \item Verification \& Benchmarking Framework
        \end{itemize}
        \item Results
        \begin{itemize}
            \item Result likelihood
            \item Performance
        \end{itemize}
        \item Conclusion
    \end{itemize}
    \centering
    \caption{Planned structure of thesis}
    \label{fig:thesisstructure}
\end{figure}

\begin{figure}
    % Modified from https://git.cs.uni-paderborn.de/syssec/templates/exposee
    \begin{ganttchart}
        [x unit=0.5cm, %extend sheets to 1cm
        ]{1}{18}
        \gantttitle[title label node/.append style={below left=7pt and -3pt}]{WEEKS:\quad1}{1}
        \gantttitlelist{2,...,18}{1} \\
        \ganttgroup{Initial Implementation}{1}{4} \\
        \ganttgroup{Verification \& Benchmarking}{2}{6} \\
        \ganttgroup{Optimization}{5}{12} \\
        \ganttgroup{Writing \& Proofreading}{13}{18}
    \end{ganttchart}
    \centering
    \caption{Worst-Case work schedule}
    \label{fig:worstschedule}
\end{figure}

According to the regulations, the workload of the thesis is supposed to contain nine weeks of full time labor, which is equivalent to 18 weeks of half time labor. The deadline for handing the thesis in is five months, which is approximately equivalent to 21 to 22 weeks and leaves a buffer of three to four weeks. We assume that the initial implementation will take approximately four weeks, as we will need to rewrite the complete application due to its heavy use of dynamic memory in datastructures and the kernel code, as well as its imperative structure. Benchmarking will be rather easy with our experience in profiling oneAPI FPGA designs, but since we have no experience in statistical verification, we can not actually make a good prediction for the required time. For writing the thesis, other students have quoted approximately four to six weeks of half-time work for thesis writing and proofreading, which leaves approximately nine to thirteen weeks between the initial implementation and writing the thesis for optimization. The resulting worst-case work schedule is visualized in Figure \ref{fig:worstschedule}.

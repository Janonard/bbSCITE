\section{Goals, Results, and Structure of the Thesis}

\begin{figure}
    % Modified from https://git.cs.uni-paderborn.de/syssec/templates/exposee
    \begin{ganttchart}
        [x unit=0.5cm, %extend sheets to 1cm
        ]{1}{18}
        \gantttitle[title label node/.append style={below left=7pt and -3pt}]{WEEKS:\quad1}{1}
        \gantttitlelist{2,...,18}{1} \\
        \ganttgroup{Initial Implementation}{1}{4} \\
        \ganttgroup{Verification \& Benchmarking}{2}{6} \\
        \ganttgroup{Optimization}{5}{12} \\
        \ganttgroup{Writing \& Proofreading}{13}{18}
    \end{ganttchart}
    \centering
    \caption{Planned worst-case work schedule}
    \label{fig:worstschedule}
\end{figure}

Our main goals for the thesis were twofold: Firstly, we set ourselves the goal to accelerate the SCITE algorithm with FPGAs. This new implementation, which we called \ac{ffSCITE}, should perform more chain steps per second than the reference implementation by Jahn et al. \cite{tree2016}, using the Intel Stratix 10 GX 2800 \acp{FPGA} and the AMD Milan 7763 \acp{CPU} of the Noctua 2 supercomputer at the Paderborn University, respectively. Secondly, we set the goal to design and apply a statistical set to verify that \ac{SCITE} and \ac{ffSCITE} produce solutions of equivalent quality. Additionally, we set ourselves the optional goals to achieve higher throughput than the optimized CPU implementation by Ernst et al. \cite{ernst2020Performance} and to adapt our implementation to the scoring model of \ac{infSCITE} published by Kuipers et al. \cite{kuipers2017single}, which is an improved version of \ac{SCITE}. We planned to work with the worst-case schedule outlined in figure \ref{fig:worstschedule}.

The final build of \ac{ffSCITE} achieves a constant throughput of 566.72 thousand chain steps per second, which is a speed up of up to 8.63 compared to \ac{SCITE}. This matches our performance model which predicted that the design requires 512 cycles to process one step at the clock speed of 295.83 MHz. Our quality test is based on the \ac{TOST} procedure \cite{schuirmann1987comparison} and we were able to show that \ac{ffSCITE} is equivalent to \ac{SCITE} with a significance level of 2\%. However, we assume that the implementation by Ernst et al. \cite{ernst2020Performance} has higher throughput than \ac{ffSCITE} and we were not able to evaluate the extensibility of \ac{ffSCITE} to \ac{infSCITE}. We also had to remove some features of \ac{SCITE} in \ac{ffSCITE} in order to reach our goal in time, for example collecting all co-optimal trees and the $\beta$ search.

The rest of this thesis is structured as follows: First, we provide a detailed and formal overview of \ac{SCITE} and fundamental concepts of \ac{FPGA} designs in chapter \ref{ch:background}. In the following chapter \ref{ch:design}, we assume that readers are familiar with the concepts introduced in chapter \ref{ch:background} and explain the design decisions that lead to the final design of \ac{ffSCITE}. Lastly, we evaluate the different properties of \ac{ffSCITE} in chapter \ref{ch:evaluation}, namely the hardware resource usage, the throughput benchmark, and the quality test, and summarize all contributions in chapter \ref{ch:conclusion}. This chapter also contains a list of smaller, optional features that we did not implement and open directions for further optimization and analysis.
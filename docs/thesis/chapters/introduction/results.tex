\section{Goals, Results and Structure of the Thesis}

\todo[inline]{Insert the correct performance figures}

The goal for the thesis was to accelerate the SCITE algorithm with FPGAs. This new implementation, which we called \ac{ffSCITE}, should perform more chain steps per second than the initial implementation without a loss of solution quality. ffSCITE uses the Intel Stratix 10 GX 2800 FPGAs of the Noctua 2 supercomputer at the Paderborn University and executes up to XY chain steps per second, which is XY as much as SCITE executes at comparable loads. Additionally, we were also able to verify that the solutions found by \ac{SCITE} and \ac{ffSCITE} are equivalent using the \ac{TOST} \cite{schuirmann1987comparison} procedure. We had also set ourselves the optional goals to beat the optimized CPU version of \ac{SCITE} by Ernst et al. \cite{ernst2020Performance} and adapt our implementation to \ac{SCITE}'s successor \ac{infSCITE} \cite{kuipers2017single}. However, we were not able to achieve or verify this due to a lack of time.

The rest of this thesis is structured as follows: First, we will describe the original \ac{SCITE} implementation in detail, which was also the initial state of \ac{ffSCITE}. Then, we describe the final design of \ac{ffSCITE} and point out the differences to \ac{SCITE}. Some differences require more detailed descriptions and are therefore then. Lastly, we present the design and the results of the quality and performance benchmarks and list open directions that could improve \ac{ffSCITE} even further.
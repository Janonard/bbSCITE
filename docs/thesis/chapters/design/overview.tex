\section{Design Overview}
\label{sec:overview}

Our implementation of \ac{SCITE} is a multi-kernel design. We initially tried to implement the algorithm with a single kernel, but we soon encountered feedback problems that we were not able to resolve inside the single kernel. Therefore, we split the parts of the algorithm with problematic feedback into separate kernels. Figure \ref{fig:design_schematic} gives an overview of the design: The change proposer kernel receives the current state from the IO kernel and samples all random parameters of a chain step. We have tried to do as little deterministic work in the change proposer kernel to limit the effect of feedback on the design's performance. Details on this problem are discussed in section \ref{sec:move_proposal}. Everything else, like finding the deterministic parameters of the move, computing the resulting tree, computing its likelihood, and evaluating it is done by the ``Tree Scorer'' kernel. We also made some interesting optimizations on the computation of the resulting tree and its likelihood score, which we would describe as our biggest accomplishments and are described in section \ref{sec:encoding} and \ref{sec:scoring}, respectively.

\begin{figure}
    \centering
    \begin{tikzpicture}
        \draw[every rectangle node/.style={draw}]
            node (IOKernel) at (0,0) {IO Kernel}
            node (ChangeProposer) at (3,2) {Change Proposer}
            node (TreeScorer) at (3,-2) {Tree Scorer}
            node (InitialStates) at (-6.5,0) {Initial States}
            node (BestStates) at (-6.5,-4) {Best State};

        \draw[-{Stealth[length=.2cm]}] (IOKernel) 
            |- node[above,text width=2.5cm] (IOCPLink) {current tree \& score}
            (ChangeProposer);
        \draw[-{Stealth[length=.2cm]}] (ChangeProposer) 
            -- node[right,text width=2.5cm] (CPTSLink) {current tree \& score, move metadata}
            (TreeScorer);
        \draw[-{Stealth[length=.2cm]}] (TreeScorer) 
            -| node[below,text width=2.5cm] (TSIOLink) {current tree \& score}
            (IOKernel);
        \draw[-{Stealth[length=.2cm]}] (InitialStates) 
            -- node[above,text width=2.5cm] {initial tree \& score}
            (IOKernel);
        \draw[-{Stealth[length=.2cm]}] (TreeScorer) 
            |- node[below] {best tree \& score}
            (BestStates);

        \node[draw,ultra thick,fit=(IOKernel) (IOCPLink) (ChangeProposer) (CPTSLink) (TreeScorer) (TSIOLink),inner sep=.4cm] (FPGA) {};
        \node[above] at (FPGA.north) {FPGA chip design};

        \node[draw,ultra thick,fit=(InitialStates) (BestStates),inner sep=.4cm] (DDR) {};
        \node[above] at (DDR.north) {Global DDR memory};
    \end{tikzpicture}
    \caption{Schematic of the \ac{ffSCITE} chip design.}
    \label{fig:design_schematic}
\end{figure}
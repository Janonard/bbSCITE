\section{Move Proposals}
\label{sec:move_proposal}

\begin{itemize}
    \item SCITE uses \texttt{rand} syscall. Not available on FPGAs, undefined characteristics.
    \item Pseodorandom framework of C++ standard library: URNG and distribution classes.
    \item Minstd0-URNG: Minimal standard as advertised by author, implemented in C++ standard library and OneDPL.
    \item Initial plan: start with minstd0, write URNG-agnostic code, research better alternative
    \item Research not possible in time, State of the art not easily identifiable by outsider
    \begin{itemize}
        \item Every URNG has drawbacks (obviously)
        \item Set of URNGs usable for MCMC very small (most research done on cryptographic URNGs)
        \item Set of URNGs usable on FPGA also very small
        \item No intersection found, little to no implementations for HLS designs.
    \end{itemize}
    \item Further research efforts appeared to be very time consuming, the quality test was already successful
    \item Decision: Work with minstd0 and leave the issue be.
    \item Minstd0 is linear congruential engine, $X_{n+1} = (a \cdot X_n) \mod m$.
    \item Problem: Implementation uses 64-bit integer. 64-bit integer is \emph{very} resource intensive. Every call to URNG creates one instance.
    \item Random draws make up XX\% of the final design, independently of input size.
    \item Next state of URNG is it's output, which is supposed to appear random: Next state unpredictable, random draws barely pipelinable.
    \item Necessary: Reduce random draws to a minimum, move them into own kernel, make kernel loop as short as possible.
\end{itemize}
\section{Tree encoding and operations}
\label{sec:encoding}

Our first and most impactful contribution is an improvement to the used mutation tree encoding and the operations on this code. The original \ac{SCITE} implementation \cite{tree2016} uses a parent vector as the canonical data structure to encode a mutation tree:

\begin{definition}[Parent vector, \cite{tree2016}]
    \label{def:parent_vector}
    Let $T = (V, E, r)$ be a mutation tree. The corresponding parent vector is defined as the sequence $(p_v)_{v \in V} \subseteq V$ with
    \begin{align*}
        p_v &:= \begin{cases}
            p_T(v) & v \neq r \\
            r & v = r 
        \end{cases}
    \end{align*}
\end{definition}

Using this encoding has the obvious advantage that most of the tree moves are simple: The updates in the ``prune and reattach'' and ``swap subtrees'' moves of \textsc{ChainStep} (Algorithm \ref{alg:scite-step}) are mere constant operations. The update operations in the ``swap nodes'' move are more involved since every edge needs to be visited and checked, but it is still in linear runtime. However, sampling moves and computing the likelihood function requires many connectivity queries: Both the ``prune and reattach'' and the ``swap subtrees'' moves may need to sample a target that is not a descendant of the moved node, and the induced mutation matrix (Definition \ref{def:induced_mutmatrix}) is defined by node connectivity. Therefore, a separate data structure is used to answer these queries quickly:

\begin{definition}[Ancestor matrix, \cite{tree2016}]
    \label{def:ancestor_matrix}
    Let $T = (V, E, r)$ be a mutation tree. The corresponding ancestor matrix is defined as the matrix $A \in \{0,1\}^{|V| \times |V|}$ with
    \begin{align*}
        A_{v,w} &:= \begin{cases}
            1 & v \leadsto_T w \\
            0 & \text{else}
        \end{cases}
    \end{align*}
    for all $v, w \in V$.
\end{definition}

Jahn et al. \cite{tree2016} also give an algorithm that constructs an ancestor matrix from a parent vector; We have listed it as algorithm \ref{alg:ancestor_matrix}. Intuitively, it walks up from every node to the root and marks all nodes it encounters as ancestors. \ac{SCITE} uses this algorithm once per chain step to sample descendants and non-descendants for a move. Hardware implementations of this algorithm are however inefficient since it is hard to predict how often the inner while-loop is executed. For example, it may be executed $|V|$ times for the leaf of a completely degenerated tree, but it may not be executed at all for the root $r$. Therefore, the outer for-loop needs to be executed serially, which severely limits the performance of the design. However, we were able to eliminate the need to construct ancestor matrices on the device. First of all, we were able to show with theorem \ref{theo:am_reverse} that it is possible to find a node's parent using an ancestor matrix. Ancestor matrices can therefore be used as the canonical data structure to encode mutation trees. More importantly, however, we were able to show with theorems \ref{theo:swap_nodes_correctness}, \ref{theo:prune_reattach_correctness}, \ref{theo:swap_unrelated_correctness} and \ref{theo:swap_related_correctness} that every move of the \textsc{ChainStep} algorithm (Algorithm \ref{alg:scite-step}) can be executed on an ancestor matrix with linear time and space requirements. This leaves us with one linear, perfectly pipelinable loop to update the ancestor matrix for a move. Compared to the two quadratic, barely pipelinable loops to compute the ancestor matrix twice, this is certainly an improvement.

\begin{algorithm}
    \begin{algorithmic}[1]
        \Function{AncestorMatrix}{$(p_v)_{v \in V} \subseteq V, r \in V$} \Comment $r$ is the root of the tree.
            \State $n \leftarrow |V|$
            \State $A \leftarrow 0 \in \{0,1\}^{n \times n}$
            \ForAll{$w \in V$}
                \State $v \leftarrow w$
                \While{$v \neq r$}
                    \State $A_{v, w} \leftarrow 1$
                    \State $v \leftarrow p_v$
                \EndWhile
                \State $A_{r, w} \leftarrow r$
            \EndFor
            \State \Return $A$
        \EndFunction
    \end{algorithmic}
    \caption{Algorithm to construct an ancestor matrix (Definition \ref{def:ancestor_matrix}) from a parent vector (Definition \ref{def:parent_vector}), \cite{tree2016}}
    \label{alg:ancestor_matrix}
\end{algorithm}

\subsection{Reversing the ancestor matrix construction}

\begin{theorem}
    \label{theo:am_reverse}
    Let $T=(V, E, r)$ be a mutation tree, $A \in \{0, 1\}^{|V| \times |V|}$ be the corresponding ancestor matrix, and $(v, w) \in V^2$ be a node pair. The algorithm \textsc{IsParent} (Algorithm \ref{alg:is_parent}) returns True iff $v = p_T(w)$ given the inputs $V$, $A$ and $(v,w)$.
\end{theorem}

\begin{algorithm}
    \begin{algorithmic}[1]
        \Function{IsParent}{$V$, $A \in \{0,1\}^{|V| \times |V|}$, $(v, w) \in V^2$}
            \If{$A_{v,w} = 0$}
                \State \Return False
            \EndIf
            \If{$v = w$}
                \State \Return $v = r$ \Comment Per convention, the root is the only node that is also its parent.
            \EndIf
            \ForAll{$x \in V \setminus \{w\}$} \Comment Unroll completely
                \If{$A_{x,w} \neq A_{x,v}$}
                    \State \Return False
                \EndIf
            \EndFor
            \State \Return True
        \EndFunction
    \end{algorithmic}
    \caption{Algorithm to query whether an edge exists in a tree, using an ancestor matrix}
    \label{alg:is_parent}
\end{algorithm}

\begin{proof}
    The algorithm evaluates the right-hand part of the lemma \ref{lem:am_reverse} for non-root nodes to check whether the edge exists in the graph or not. Per convention, we also assume that $(r,r) \in E$, which the algorithm checks.
\end{proof}

\begin{lemma}
    \label{lem:am_reverse}
    Let $T = (V, E)$ be a tree and $v, w \in V$. We have:
    \begin{align*}
        (v, w) \in E \Leftrightarrow (\forall x \in V \setminus \{w\}: x \leadsto_T v \Leftrightarrow x \leadsto_T w) \wedge (v \leadsto_T w)
    \end{align*}
\end{lemma}

\begin{proof}
    We first show $\Rightarrow$: We obviously have $v \leadsto_T w$. Let $x \in V \setminus \{v\}$. Then, we have:
    \begin{align*}
        x \leadsto_T w  &\Rightarrow \exists p = (x, \dots, w) \subseteq E \\
                        &\stackrel{(v, w) \in E}{\Rightarrow} (v, w) \in p \\
                        &\Rightarrow p' := p \setminus \{(v, w)\} = (x, \dots, v) \subseteq E \\
                        &\Rightarrow x \leadsto_T v \\
        x \leadsto_T v  &\Rightarrow \exists p = (x, \dots, v) \subseteq E \\
                        &\stackrel{(v, w) \in E}{\Rightarrow} p' := p \cup \{(v, w)\} = (x, \dots, v, w) \subseteq E \\
                        &\Rightarrow x \leadsto_T w
    \end{align*}
    Now, we show $\Leftarrow$: Let's assume for a contradiction that we have $(\forall x \in V \setminus \{w\}: x \leadsto_T v \Leftrightarrow x \leadsto_T w) \wedge (v \leadsto_T w)$ and $(v, w) \notin E$. We have:
    \begin{align*}
        v \leadsto_T w \wedge (v, w) \notin E &\Rightarrow \exists y \in V \setminus \{v, w\}: v \leadsto_T y \leadsto_T w \\
        y \leadsto_T w \wedge y \neq w &\Rightarrow y \leadsto_T v \\
        &\Rightarrow y \leadsto_T v \leadsto_T y
    \end{align*}
    This means that our tree has a circle, which is a contradiction. Therefore, such a $y$ can not exist and we have in fact $p = (v, \dots, w) = (v, w) \Rightarrow (v, w) \in E$.
\end{proof}

\subsection{``Swap nodes'' ancestor matrix computation}

\begin{definition}[Swap nodes move, \cite{tree2016}]
    \label{def:swap_nodes}
    Let $T = (V, E, r)$ be a mutation tree, and $v, w \in V \setminus \{r\}$ with $v \neq w$. We define the mutation tree $T'$ after the ``swap nodes'' move as $T' = (V, E', r)$ with
    \begin{align*}
        E' := \{(f(x), f(y)) : (x, y) \in E\}
    \end{align*}
    We assume these definitions for the remainder of this subsection.
\end{definition}

\begin{theorem}[Correctness of the ``Swap nodes'' update]
    \label{theo:swap_nodes_correctness}
    Let $A \in \{0,1\}^{|V| \times |V|}$ be the corresponding ancestor matrix of $T$. The ancestor matrix returned by \textsc{AM::SwapNodes}$(V, A, v, w)$ (Algorithm \ref{alg:am_swap_nodes}) is the corresponding ancestor matrix of $T'$.
\end{theorem}

\begin{algorithm}
    \begin{algorithmic}[1]
        \Function{AM::SwapNodes}{$V$, $A \in \{0,1\}^{|V| \times |V|}$, $v$, $w$}
            \State $A' \leftarrow 0 \in \{0,1\}^{|V| \times |V|}$
            \ForAll{$x \in V$}
                \If{$x = v$}
                    \State $A'[x] \leftarrow A[w]$
                \ElsIf{$x = w$}
                    \State $A'[x] \leftarrow A[v]$
                \Else
                    \State $A'[x] \leftarrow A[x]$
                \EndIf
                \State $A'[x][v], A'[x][w] \leftarrow A'[x][w], A'[x][v]$ \Comment Bit Swap
            \EndFor
            \State \Return $A'$
        \EndFunction
    \end{algorithmic}
    \caption{Algorithm to perform the ``swap nodes'' move on an ancestor matrix. All edges from and to $v$ are $w$ are swapped, assuming that we have $v \neq w$.}
    \label{alg:am_swap_nodes}
\end{algorithm}

\begin{proof}
    Let $A'$ be the return value of the mentioned function call. We need to show that $A'$ has the correct value for every node pair $x, y \in V$. To be precise, we need to show that $A'[x][y] = f(x) \leadsto_T f(y)$ for all $x, y \in V$ since we have $x \leadsto_{T'} y \Leftrightarrow f(x) \leadsto_T f(y)$ with Lemma \ref{lem:swap_nodes_property}. We distinguish several cases for the node pair:
    \begin{itemize}
        \item $x \notin \{v, w\} \wedge y \notin \{v, w\}$. In this case, the algorithm initially sets $A'[x]$ to $A[x]$ and the $y$-th bit is untouched by the bit swap, so we have $A'[x][y] = A[x][y] = A[f(x)][f(y)] = f(x) \leadsto_T f(y)$.
        
        \item $x \in \{v, w\} \wedge y \notin \{v, w\}$. We show the sub-case $x = v$ since $x = w$ is analogous. The algorithm initially sets $A'[x]$ to $A[w]$ and the $y$-th bit is untouched by the bit swap. Therefore, we have $A'[x][y] = A[w][y] = A[f(x)][f(y)] = f(x) \leadsto_T f(y)$.
        
        \item $x \notin \{v, w\} \wedge y \in \{v, w\}$. We show the sub-case $y = v$ since $y = w$ is analogous. The algorithm initially sets $A'[x]$ to $A[x]$ and swaps the $v$-th bit with the $w$-th bit. Therefore, we have $A'[x][y] = A[x][w] = A[f(x)][f(y)] = f(x) \leadsto_T f(y)$.
        
        \item $x \in \{v, w\} \wedge y \in \{v, w\} \wedge x = y$. We show the sub-case $x = y = v$ since $x = y = w$ is analogous. The algorithm initially sets $A'[x]$ to $A[w]$ and swaps the $v$-th bit with the $w$-th bit. Therefore, we have $A'[x][y] = A[w][w] = A[f(x)][f(y)] = f(x) \leadsto_T f(y)$.
        
        \item $x \in \{v, w\} \wedge y \in \{v, w\} \wedge x \neq y$. We show the sub-case $x = v$ since $x = w$ is analogous. We have $x = v \wedge x \neq y \Rightarrow y = w$. The algorithm initially sets $A'[x]$ to $A[w]$ and swaps the $v$-th bit and the $w$-th bit. Therefore, we have $A'[x][y] = A[w][v] = A[f(x)][f(y)] = f(x) \leadsto_T f(y)$.
    \end{itemize}
\end{proof}

\begin{lemma}
    \label{lem:swap_nodes_property}
    We have $x \leadsto_{T'} y \Leftrightarrow f(x) \leadsto_T f(y)$ for all $x, y \in V$.
\end{lemma}

\begin{proof}
    First, it should be noted that $f$ is obviously self-inverse, so that we have $f(f(x)) = x$ and $f(f(T)) = T$. We therefore only need to show $x \leadsto_{T} y \Rightarrow f(x) \leadsto_{T'} f(y)$ since the rest follows. We have:
    \begin{align*}
        x \leadsto_T y &\Rightarrow \exists p = (x, \dots, y) = \{(x, p_2), (p_2, p_3), \dots, (p_{l-1}, y)\} \subseteq E \\
        &\Rightarrow p' = \{(f(x), f(p_2)), (f(p_2), f(p_3)), \dots, (f(p_{l-1}), f(y))\} \subseteq E' \\
        &\Rightarrow f(x) \leadsto_{T'} f(y)
    \end{align*}
\end{proof}

\subsection{``Prune and reattach'' ancestor matrix computation}

\begin{definition}[``Prune and reattach'' move, \cite{tree2016}]
    \label{def:prune_and_reattach}
    Let $T = (V, E, r)$ be a mutation tree, $v \in V \setminus \{r\}$ and $t \in V$ with $v \not\leadsto_T t$. We define the mutation tree $T'$ after the ``prune and reattach'' move as $T' = (V, E', r)$ with
    \begin{align*}
        E' := (E \setminus \{(p_T(v), v)\}) \cup \{(t, v)\}
    \end{align*}
    We assume these definitions for the remainder of this subsection.
\end{definition}

\begin{theorem}[Correctness of the ``Prune and Reattach'' update]
    \label{theo:prune_reattach_correctness}
    Let $A \in \{0,1\}^{|V| \times |V|}$ be the corresponding ancestor matrix of $T$. The ancestor matrix returned by \textsc{AM::Prune\-Reattach}\-$(V, A, v, t)$ (Algorithm \ref{alg:am_prune_reattach}) is the corresponding ancestor matrix of $T'$.
\end{theorem}

\begin{algorithm}
    \begin{algorithmic}[1]
        \Function{AM::PruneReattach}{$V$, $A \in \{0, 1\}^{|V| \times |V|}$, $v$, $t$}
            \State $A' \leftarrow 0 \in \{0,1\}^{|V| \times |V|}$
            \ForAll{$x \in V$}
                \ForAll{$y \in V$} \Comment Unroll completely
                    \If{$A[v][y]$}
                        \State $A'[x][y] \leftarrow A[x][t] \vee (A[v][x] \wedge A[x][y])$ \Comment See lemma \ref{lem:prune_reattach_rel}
                    \Else
                        \State $A'[x][y] \leftarrow A[x][y]$ \Comment See lemma \ref{lem:prune_reattach_unrel}
                    \EndIf
                \EndFor
            \EndFor
            \State \Return $A'$
        \EndFunction
    \end{algorithmic}
    \caption{Algorithm to perform the ``prune and reattach'' move on an ancestor matrix. The node $v$ is attached to the node $t$, assuming that we have $v \not\leadsto_T t$.}
    \label{alg:am_prune_reattach}
\end{algorithm}

\begin{proof}
    This theorem follows directly from lemmata \ref{lem:prune_reattach_unrel} and \ref{lem:prune_reattach_rel}.
\end{proof}

\begin{remark}
    \label{rem:prune_reattach}
    We remark that we have $v \leadsto_T y \Leftrightarrow v \leadsto_{T'} y$ for all $y \in V$ and that we have $E' \setminus E = \{(t,v)\}$ as well as $E \setminus E' = \{(p_T(v), v)\}$.
\end{remark}

The idea behind $v \leadsto_T y \Leftrightarrow v \leadsto_{T'} y$ is simply that children of $v$ still stay connected to $v$ after the move. $E' \setminus E = \{(t,v)\}$ as well as $E \setminus E' = \{(p_T(v), v)\}$ can be computed manually.

\begin{lemma}
    \label{lem:prune_reattach_unrel}
    Let $x, y \in V$ with $v \not\leadsto_T y$. We have:
    \begin{align*}
        x \leadsto_{T'} y \Leftrightarrow x \leadsto_T y
    \end{align*}
\end{lemma}

The idea behind this lemma is that paths in the tree are unharmed if they do not go via the removed edge $(p_T(v), v)$.

\begin{proof}
    Let's first assume $x \leadsto_{T'} y$ and $x \not\leadsto_{T} y$ to prove $x \leadsto_{T'} y \Rightarrow x \leadsto_T y$ by contradiction. There is a path $p = (p_1, \dots, p_l) \subseteq E'$ with $p_1 = x$ and $p_l = y$. We have $p \not\subseteq E$ since we would otherwise have $x \leadsto_T y$. This means that there is an edge $e \in (p \setminus E) \subseteq (E' \setminus E)$. One can easily check that $(t, v)$ is the only edge in $E' \setminus E$. We therefore have $e = (t, v)$ and $\exists i \leq l: p_i = v$. This however means that $p' := (v, p_{i+1}, \dots, y)$ is a suffix of $p$. Since $e = (t,v) \notin p'$, we have $p' \subseteq E \Rightarrow v \leadsto_{T} y$, which contradicts our precondition of $v \not\leadsto_T y$.

    Using an analogous argumentation with the assumption of $x \not\leadsto_{T'} y$ and $x \leadsto_T y$ leads to $v \leadsto_{T'} y$. From this and remark \ref{rem:prune_reattach}, $v \leadsto_T y$ follows. This is the same contradiction as above and proves $x \leadsto_T y \Rightarrow x \leadsto_{T'} y$.
\end{proof}

\begin{lemma}
    \label{lem:prune_reattach_rel}
    Let $x, y \in V$ with $v \leadsto_T y$. We have:
    \begin{align*}
        x \leadsto_{T'} y \Leftrightarrow (v \leadsto_T x \leadsto_T y \vee x \leadsto_T t)
    \end{align*}
\end{lemma}

The idea of this lemma is that paths over the node $v$ exist in the new tree iff they are either entirely in the subtree below $v$ or are newly established via the edge $(t, v)$.

\begin{proof}
    We have nothing to show for $x = v$ due to remark \ref{rem:prune_reattach}. Therefore, we can assume $x \neq v$ in the following. We first show $\Leftarrow$: If we have $x \leadsto_T t$, we also have $x \leadsto_{T'} t$ with $v \not\leadsto_T t$ and lemma \ref{lem:prune_reattach_unrel}. From this and $(t, v) \in E' \Rightarrow t \leadsto_{T'} v$, it follows that we have $x \leadsto_{T'} t \leadsto_{T'} v \leadsto_{T'} y \Rightarrow x \leadsto_{T'} y$.

    We now show $\Rightarrow$: Let $p = (p_1, \dots, p_l) \subseteq E'$ be the path from $x$ to $y$ in $T'$ ($p_1 = x$ and $p_l = y$). If $(t, v) \in p$, we then have $x \leadsto_{T'} t$, and with $v \not\leadsto_T t$ and lemma \ref{lem:prune_reattach_unrel} also $x \leadsto_T t$. If we have $(t, v) \notin p$, we also have $x \not\leadsto_{T'} v$. From this, we need to conclude $v \leadsto_{T'} x \leadsto_{T'} y$ in order to avoid a contradiction with $v \leadsto_{T'} y$. Lastly, we have $v \leadsto_T x \leadsto_T y$ with remark \ref{rem:prune_reattach} and since we neither have $(v, t) \in p$ nor $(p_T(v), v) \in p$.
\end{proof}

\subsection{``Swap unrelated subtrees'' ancestor matrix computation}

\begin{definition}[``Swap unrelated subtrees'' move, \cite{tree2016}]
    \label{def:swap_unrelated_subtrees}
    Let $T = (V, E, r)$ be a mutation tree, and $v, w \in V \setminus \{r\}$ with $v \neq w$. We define the mutation tree $T'$ after the ``swap unrelated subtrees'' as $T' = (V, E', r)$ with
    \begin{align*}
        E' := (E \setminus \{(p_T(v), v), (p_T(w), w)\}) \cup \{(p_T(w), v), (p_T(v), w)\}
    \end{align*}
    We assume these definitions for the remainder of this subsection.
\end{definition}

\begin{theorem}[Correctness of the ``Swap unrelated Subtrees'' update]
    \label{theo:swap_unrelated_correctness}
    Let $A \in \{0,1\}^{|V| \times |V|}$ be the corresponding ancestor matrix of $T$. The ancestor matrix returned by \textsc{AM::Swap\-Unrelated\-Subtrees}$(V, A, v, w)$ (Algorithm \ref{alg:am_swap_unrelated}) is the corresponding ancestor matrix of $T'$.
\end{theorem}

\begin{algorithm}
    \begin{algorithmic}[1]
        \Function{AM::SwapUnrelatedSubtrees}{$V$, $A \in \{0, 1\}^{|V| \times |V|}$, $v$, $w$}
        \State $A' \leftarrow 0 \in \{0,1\}^{|V| \times |V|}$
        \ForAll{$x \in V$}
            \ForAll{$y \in V$} \Comment Unroll completely
                \If{$A[v][y] \wedge \neg A[w][y]$}
                    \State $A'[x][y] \leftarrow A[x][p_T(w)] \vee (A[v][x] \wedge A[x][y])$
                \ElsIf{$\neg A[v][y] \wedge A[w][y]$}
                    \State $A'[x][y] \leftarrow A[x][p_T(v)] \vee (A[v][x] \wedge A[x][y])$
                \Else
                    \State $A'[x][y] \leftarrow A[x][y]$
                \EndIf
            \EndFor
        \EndFor
        \State \Return $A'$
        \EndFunction
    \end{algorithmic}
    \caption{Algorithm to perform the ``swab subtrees'' move for unrelated subtrees on an ancestor matrix. The node $v$ is attached to $p_T(w)$ and the node $w$ is attached to $p_T(v)$, assuming that we have $v \neq w$, $v \not\leadsto_T w$, and $w \not\leadsto_T v$.}
    \label{alg:am_swap_unrelated}
\end{algorithm}

\begin{proof}
    First of all, we can remark that the ``Swap unrelated subtrees'' move is equivalent to two ``Prune and Reattach'' moves, first with parameters $v, p_T(w)$ and then with parameters $w, p_T(v)$ ($p_T(v)$ represents the parent of $v$ \textit{before} the first move). Therefore, we can prove the correctness by looking at every pair $x,y \in V$ and showing that \textsc{AM::SwapUnrelatedSubtrees} \ref{alg:am_swap_unrelated} assigns the same values as two \textsc{AM::PruneReattach} \ref{alg:am_prune_reattach} calls would. Like the algorithm, we consider the following cases:
    \begin{itemize}
        \item $v \leadsto_T y$ and $w \not\leadsto_T y$: The first call $A' \leftarrow \textsc{AM::PruneReattach}(V, A, v, p_T(w))$ assigns 
        \begin{align*}
            A'[x][y] \leftarrow A[x][p_T(w)] \vee (A[v][x] \wedge A[x][y])
        \end{align*}
        and the second call $A'' \leftarrow \textsc{AM::PruneReattach}(V, A', w, p_T(v))$ assigns $A''[x][y] \leftarrow A'[x][y]$. \textsc{AM::SwapUnrelatedSubtrees} is therefore correct in this case.
        \item $v \not\leadsto_T y$ and $w \leadsto_T y$: This case is analogous to the previous one.
        \item $v \not\leadsto_T y$ and $w \not\leadsto_T y$: Both calls to \textsc{AM::PruneReattach} forward the previous value of $A[x][y]$, just like \textsc{AM::SwapUnrelatedSubtrees} does.
        \item $v \leadsto_T y$ and $w \leadsto_T y$: This case is not possible since it would lead to either $v \leadsto_T w \leadsto_T y$ or $w \leadsto_T v \leadsto_T y$, which would contradict our assumption of $v \not\leadsto_T w$ and $w \not\leadsto_T v$. It is therefore safe for \textsc{AM::SwapUnrelatedSubtrees} to not catch this case.
    \end{itemize}
    \textsc{AM::SwapUnrelatedSubtrees} is therefore correct in all cases.
\end{proof}

\subsection{``Swap related subtrees'' ancestor matrix computation}

\begin{definition}[``Swap related subtrees'' move, \cite{tree2016}]
    \label{def:swap_related_subtrees}
    Let $T = (V, E, r)$ be a mutation tree, $v, w \in V \setminus \{r\}$ with $v \neq w$ and $w \leadsto_T v$, and $t \in V$ with $v \leadsto_T t$. We define the mutation tree $T'$ after the ``swap related subtrees'' move as $T = (V, E', r)$ with
    \begin{align*}
        E' := (E \setminus \{(p_T(v), v), (p_T(w), w)\}) \cup \{(p_T(w), v), (t, w)\}
    \end{align*}
    We assume these definitions for the remainder of this subsection.
\end{definition}

\begin{theorem}[Correctness of the ``swap related subtrees'' update]
    \label{theo:swap_related_correctness}
    Let $A \in \{0,1\}^{|V| \times |V|}$ be the corresponding ancestor matrix of $T$. The ancestor matrix returned by \textsc{AM::Swap\-Related\-Subtrees}$(V, A, v, w, t)$ (Algorithm \ref{alg:am_swap_related}) is the corresponding ancestor matrix of $T'$.
\end{theorem}

\begin{algorithm}
    \begin{algorithmic}[1]
        \Function{Classify}{$A, v, w, x$}
            \If{$A[v][x]$}
                \State \Return 2
            \ElsIf{$A[w][x]$}
                \State \Return 1
            \Else
                \State \Return 0
            \EndIf
        \EndFunction
        \State
        \Function{AM::SwapRelatedSubtrees}{$V$, $A \in \{0, 1\}^{|V| \times |V|}$, $v$, $w$, $t$}
        \State $A' \leftarrow 0 \in \{0,1\}^{|V| \times |V|}$
        \ForAll{$x \in V$}
            \State $c_x \leftarrow \textsc{Classify}(A, v, w, x)$

            \ForAll{$y \in V$} \Comment Unroll completely
                \State $c_y \leftarrow \textsc{Classify}(A, v, w, y)$

                \If{$(c_x = c_y) \vee (c_x = 0 \wedge c_y \neq 0)$}
                    \State $A'[x][y] \leftarrow A[x][y]$
                \ElsIf{$c_x = 2 \wedge c_y = 1$}
                    \State $A'[x][y] \leftarrow A[x][t]$
                \Else
                    \State $A'[x][y] \leftarrow 0$
                \EndIf
            \EndFor
        \EndFor
        \State \Return $A'$
        \EndFunction
    \end{algorithmic}
    \caption{Algorithm to perform the ``swab subtrees'' move for related subtrees on an ancestor matrix. The node $v$ is attached to $p_T(w)$ and the node $w$ is attached to $t$, assuming that we have $v \neq w$ and $w \leadsto_T v$.}
    \label{alg:am_swap_related}
\end{algorithm}

\begin{definition}
    \label{def:related_swap_classes}
    We partition $V$ in the following classes:
    \begin{align*}
      A &= \{x \in V: v \not\leadsto_T x \wedge w \not\leadsto_T y\} \\
      B &= \{x \in V: v \not\leadsto_T x \wedge w \leadsto_T x\} \\
      C &= \{x \in V: v \leadsto_T x\}
    \end{align*}
\end{definition}

% Adapted from https://tex.stackexchange.com/questions/37462/placing-a-triangle-around-nodes-in-a-tree.
\pgfmathsetmacro{\sinOffset}{sin(60)}
\pgfmathsetmacro{\cosOffset}{cos(60)}

\begin{figure}
    \centering
    \begin{tikzpicture}
        \node (r1) {$r$}[level distance=1cm] child {
            node (pw1) {$p_T(w)$}[level distance=2cm] child [->] {
                node (w1) {$w$}[level distance=1cm] child {
                    node (pv1) {$p_T(v)$}[level distance=2cm] child {
                        node (v1) {$v$}[level distance=1cm] child {
                            node (t1) {$t$}
                            edge from parent [decorate,decoration={snake,amplitude=0.05cm},->]
                        }
                        child {
                            node (cn1) {}
                            edge from parent [decorate,decoration={snake,amplitude=0.05cm},->]
                        }
                    }
                    edge from parent [decorate,decoration={snake,amplitude=0.05cm},->]
                }
                child {
                    node (bn1) {}
                    edge from parent [decorate,decoration={snake,amplitude=0.05cm},->]
                }
            }
            edge from parent [decorate,decoration={snake,amplitude=0.05cm},->]
        }
        child {
            node (an1) {}
            edge from parent [decorate,decoration={snake,amplitude=0.05cm},->]
        };

        \draw[thick] ($(r1) + (0,1)$) -- ($(pw1) + (-\sinOffset,-\cosOffset)$) -- ($(an1) + (\sinOffset,-\cosOffset)$) -- cycle;
        \draw[thick] ($(w1) + (0,1)$) -- ($(pv1) + (-\sinOffset,-\cosOffset)$) -- ($(bn1) + (\sinOffset,-\cosOffset)$) -- cycle;
        \draw[thick] ($(v1) + (0,1)$) -- ($(t1) + (-\sinOffset,-\cosOffset)$) -- ($(cn1) + (\sinOffset,-\cosOffset)$) -- cycle;

        \node[thick] at ($(r1) + (-1,0)$) {$A$};
        \node[thick] at ($(w1) + (-1,0)$) {$B$};
        \node[thick] at ($(v1) + (-1,0)$) {$C$};

        \node at (6,0) (r2) {$r$}[level distance=1cm] child {
            node (pw2) {$p_T(w)$}[level distance=2cm] child [->] {
                node (v2) {$v$}[level distance=1cm] child {
                    node (t2) {$t$}[level distance=2cm] child {
                        node (w2) {$w$}[level distance=1cm] child {
                            node (pv2) {$p_T(v)$}
                            edge from parent [decorate,decoration={snake,amplitude=0.05cm},->]
                        }
                        child {
                            node (bn2) {}
                            edge from parent [decorate,decoration={snake,amplitude=0.05cm},->]
                        }
                    }
                    edge from parent [decorate,decoration={snake,amplitude=0.05cm},->]
                }
                child {
                    node (cn2) {}
                    edge from parent [decorate,decoration={snake,amplitude=0.05cm},->]
                }
            }
            edge from parent [decorate,decoration={snake,amplitude=0.05cm},->]
        }
        child {
            node (an2) {}
            edge from parent [decorate,decoration={snake,amplitude=0.05cm},->]
        };

        \draw[thick] ($(r2) + (0,1)$) -- ($(pw2) + (-\sinOffset,-\cosOffset)$) -- ($(an2) + (\sinOffset,-\cosOffset)$) -- cycle;
        \draw[thick] ($(w2) + (0,1)$) -- ($(pv2) + (-\sinOffset,-\cosOffset)$) -- ($(bn2) + (\sinOffset,-\cosOffset)$) -- cycle;
        \draw[thick] ($(v2) + (0,1)$) -- ($(t2) + (-\sinOffset,-\cosOffset)$) -- ($(cn2) + (\sinOffset,-\cosOffset)$) -- cycle;

        \node[thick] at ($(r2) + (-1,0)$) {$A$};
        \node[thick] at ($(w2) + (-1,0)$) {$B$};
        \node[thick] at ($(v2) + (-1,0)$) {$C$};

        \draw[dashed] ($(v2) + (-3,-4.5)$) -- ($(v2) + (-3,4.5)$);
        \node[] at ($(r1) + (-2,1)$) {$T:$};
        \node[] at ($(r2) + (-2,1)$) {$T':$};
    \end{tikzpicture}
    \caption{Illustration of the different classes from definitions \ref{def:related_swap_classes} before and after the ``swap related subtrees'' move. Squiggly lines indicate connectivity, while straight lines indicate proper edges.}
    \label{fig:related_swap_classes}
\end{figure}

\todo[inline]{I needed to use code from stackexchange for figure \ref{fig:related_swap_classes}. How do I cite this?}

\begin{remark}
    \label{rem:related_swap_classes}
    Let $A$, $B$, $C$ be the classes from definition \ref{def:related_swap_classes}. 
    \begin{enumerate}
        \item We remark that we have $x \leadsto_{T'} y \Leftrightarrow x \leadsto_T y$ if $x, y \in V$ are in the same class. We also remark that we have:
        \begin{align*}
            v &\in C & w &\in B & p_T(w) &\in A & t &\in C
        \end{align*}
        \item From 1. and the definition of $E'$ (Definition \ref{def:swap_related_subtrees}), two trivial connection chains follow:
        \begin{align*}
            p_T(w) \leadsto_T w &\leadsto_T v \leadsto _T t \\
            p_T(w) \leadsto_{T'} v &\leadsto_{T'} t \leadsto_{T'} w
        \end{align*}
    \end{enumerate}
\end{remark}

\begin{lemma}
    \label{lem:related_swap_abc}
    Let $x \in A$, and $y \in B \cup C$. We have $x \leadsto_{T'} y \Leftrightarrow x \leadsto_T y$.
\end{lemma}

\begin{proof}
    We will first show $\Leftarrow$. We have $y \in B \cup C \Rightarrow w \leadsto_T y$ and $x \in A \Rightarrow w \not\leadsto_T x$. Therefore, $w$ has to be included in the path between $x$ and $y$, and we have $x \leadsto_T w \leadsto_T y$. If we have $y \in B$, we have:
    \begin{align*}
    x \leadsto_T w &\Rightarrow x \leadsto_{T'} p_T(w) \stackrel{\ref{rem:related_swap_classes}}{\leadsto_{T'}} w \stackrel{y \in B}{\leadsto_{T'}} y
    \end{align*}
    If we have $y \in C$, we have:
    \begin{align*}
        x \leadsto_T w &\Rightarrow x \leadsto_{T'} p_T(w) \leadsto_{T'} v \stackrel{y \in C}{\leadsto_{T'}} y
    \end{align*}

    We will now show $\Rightarrow$. If we have $y \in C$, we trivially have $p_T(w) \leadsto_{T'} v \leadsto_{T'} y$, and if we have $y \in B$, we have
    \begin{align*}
        p_T(w) \leadsto_{T'} v \leadsto_{T'} t \stackrel{\text{move}}{\leadsto_{T'}} w \stackrel{y \in B}{\leadsto_{T'}} y    
    \end{align*}
    We therefore always have $p_T(w) \leadsto_{T'} y$. Since $x \in A$, we have $p_T(w) \not\leadsto_{T'} x$ and therefore $x \leadsto_{T'} p_T(w) \leadsto_{T'} y$. If we have $y \in C$, we have:
    \begin{align*}
        x \leadsto_T p_T(w) \stackrel{\text{Def.}}{\leadsto_T} w \leadsto_T v \stackrel{y \in C}{\leadsto_T} y
    \end{align*}
    If we have $y \in B$, we have:
    \begin{align*}
        x \leadsto_T p_T(w) \stackrel{\text{Def.}}{\leadsto_T} w \stackrel{y \in B}{\leadsto_T} y
    \end{align*}
\end{proof}

\begin{lemma}
    \label{lem:related_swap_bac}
    If $x \in B$ and $y \in A \cup C$, we have $x \not\leadsto_{T'} y$.
\end{lemma}

\todo[inline]{Prove!}

\begin{lemma}
    \label{lem:related_swap_cb}
    If $x \in C$ and $y \in B$, we have $x \leadsto_{T'} y \Leftrightarrow x \leadsto_T t$.
\end{lemma}

\begin{proof}
    We have:
    \begin{align*}
        x \leadsto_{T'} y \wedge y \in B \wedge x \in C &\Rightarrow x \leadsto_{T'} y \wedge w \leadsto_{T'} y \wedge w \not\leadsto_{T'} x \\
        &\Rightarrow x \leadsto_{T'} w \leadsto_{T'} y \\
        &\Rightarrow x \leadsto_{T'} t \\
        &\stackrel{x, t \in C}{\Rightarrow} x \leadsto_T t \\
        x \leadsto_T t &\stackrel{x, t \in C}{\Rightarrow} x \leadsto_{T'} t \\
        &\Rightarrow x \leadsto_{T'} t \stackrel{\text{move}}{\leadsto_{T'}} w \stackrel{y \in B}{\leadsto_{T'}} y
    \end{align*}
\end{proof}

\subsection{Implementation}
\label{sec:encoding_implementation}

All algorithms introduced in the previous section rely on bit-level operations. Most prominently, the algorithms \textsc{AM::PruneReattach} \ref{alg:am_prune_reattach}, \textsc{AM::SwapUnrelatedSubtrees} \ref{alg:am_swap_unrelated} and \textsc{AM::SwapRelatedSubtrees} \ref{alg:am_swap_related} iterate over and set every individual bit in the ancestor matrix. This would be hard to implement efficiently with general-purpose \acp{CPU}, but we were able to use the special properties of \acp{FPGA} to implement these algorithms efficiently.

First of all, we constrained the maximal number of nodes to a power of two called $n_\mathrm{max}$. This number is arbitrary and is fixed to the executable once it is built, but it can be changed in the code by changing one constant. In the final build, we have used $n_\mathrm{max} = 64$ since it is the highest value that still resulted in a synthesizable design. Every ancestor matrix therefore technically contains $n_\mathrm{max} \times n_\mathrm{max}$ entries regardless of the input size and every loop over the set of nodes is executed exactly $n_\mathrm{max}$ times. Irrelevant entries are ignored and irrelevant loop iterations perform arbitrary but unharmful operations. This eliminates the possibility to shortcut loops and to improve the performance for small inputs, but it assures that the loops inside the tree scoring kernel do not get out of order. Without fixed trip counts, we could not pipeline the complete design, which is one of the core sources of performance on \acp{FPGA}. Continuing with our optimizations, we have encoded ancestor matrices as arrays of $n_\mathrm{max}$ $n_\mathrm{max}$-bit words, where the $i$th word describes the descendants of the $i$th node. Individual words can therefore easily be stored in registers and whole matrices can be stored in \ac{RAM} blocks. Therefore, the first index of an ancestor matrix access technically addresses a word in a memory block and the second index isolates an individual bit. Next, the memory access patterns of the update algorithms are very easy to predict: Given the previous ancestor matrix $A$ we only need to preload the words $A[v]$ and $A[w]$ and then load the word $A[x]$ for every iteration of the outer for-loop; All queries required by the algorithm can be executed with these three words. The \textsc{IsParent} algorithm \ref{alg:is_parent} even needs only two memory accesses in total to operate, namely $A[v]$ and $A[x]$. Lastly, all individual bit operations are independent of each other and so simple that the inner loops of \ref{alg:is_parent}, \ref{alg:am_prune_reattach}, \ref{alg:am_swap_unrelated} and \ref{alg:am_swap_related} can be fully unrolled. In the end, every update and every parent search can be described as a single loop that loads a word, executes custom, self-contained logic on it, and writes the resulting word back, if necessary. This is perfect for \acp{FPGA} since this custom logic can be implemented parallelized on the bit level, while a \ac{CPU} would need to isolate and shift bits to accomplish this. This loop is also easily pipelinable and requires exactly $n_\mathrm{max}$ cycles to complete if its latency can be neglected. As an example, we have provided algorithm listing \ref{alg:am_prune_reattach_optimized} which is an optimized version of \textsc{AM::PruneReattach} \ref{alg:am_prune_reattach}. Lastly, we also needed to merge the update algorithms \ref{alg:am_swap_nodes}, \ref{alg:am_prune_reattach}, \ref{alg:am_swap_unrelated}, and \ref{alg:am_swap_related} into one for-loop so that there is only one loop that is always executed instead of four individual loops of which only one is executed, which would inhibit the pipelining of the entire tree scoring kernel.

\begin{algorithm}
    \begin{algorithmic}[1]
        \Function{AM::PruneReattach}{$V$, $A \in \{0, 1\}^{|V| \times |V|}$, $v$, $t$}
            \State $A' \leftarrow 0 \in \{0,1\}^{|V| \times |V|}$
            \State $p_v \leftarrow A[v]$
            \ForAll{$x \in V$}
                \State $p_\mathrm{old} \leftarrow A[x]$
                \State $p_\mathrm{new} \leftarrow 0 \in \{0,1\}^{|V|}$
                \ForAll{$y \in V$} \Comment Unroll completely
                    \If{$p_v[y]$}
                        \State $p_\mathrm{new}[y] \leftarrow p_\mathrm{old}[t] \vee (p_v[x] \wedge p_\mathrm{old}[y])$
                    \Else
                        \State $p_\mathrm{new}[y] \leftarrow p_\mathrm{old}[y]$
                    \EndIf
                \EndFor
                \State $A'[x] \leftarrow p_\mathrm{new}$
            \EndFor
            \State \Return $A'$
        \EndFunction
    \end{algorithmic}
    \caption{A version of the ``prune and reattach'' algorithm \ref{alg:am_prune_reattach} using the optimizations described in section \ref{sec:encoding_implementation}. Reused data is preloaded as early as possible and direct memory accesses are moved out of the inner loop. This loop is then unrolled and forms a custom logic block.}
    \label{alg:am_prune_reattach_optimized}
\end{algorithm}